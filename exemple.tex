\documentclass[a4paper]{article}

\usepackage[T1]{fontenc}
\usepackage[francais]{babel}
\usepackage{fullpage}
\usepackage[vlined,ruled]{algorithm2e}
\usepackage{amssymb}
\usepackage{amsmath,alltt}

\newenvironment{algo}[1]{%
\vspace{1cm}
\renewcommand{\algorithmcfname}{Algorithm}
\begin{algorithm}[H]
\label{#1}
\caption{#1}
\SetKwInOut{Constant}{Constant}
\SetKwInOut{Input}{Input}
\SetKwInOut{Output}{Output}
\SetKwInOut{Global}{Global}
\SetKwInOut{Local}{Local}
}{%
\end{algorithm}
\vspace{1cm}
}


\newcommand{\true}{\mbox{\it true}}
\newcommand{\false}{\mbox{\it false}}
\newcommand{\Boolean}{\{\true,\false\}}
\newcommand{\Integer}{\mathbb{Z}}
\newcommand{\Complex}{\mathbb{C}}
\newcommand{\Real}{\mathbb{R}}

\begin{document}
\title{Projet de compilation}
\maketitle

Seul les algorithmes sont pris en compte pour la compilation. Tout le
reste du document \LaTeX \; est ignor\'e. Le point d'entr\'ee est
l'algorithme \textbf{main}.

\begin{algo}{factorielle}
\Input{$n \in \Integer$}
\Output{$accu \in \Integer$}
\Global{$\emptyset$}
\Local{$\emptyset$}
\BlankLine
  $accu \leftarrow 1$ \;
  \While{$n \neq 0$} {
    $accu \leftarrow n \times accu$ \;
    $n \leftarrow n - 1$ \;
  }
\end{algo}

Du blabla pas pris en compte dans la compilation...

\begin{algo}{factorielleRec}
\Input{$n \in \Integer$}
\Output{$accu \in \Integer$}
\Global{$\emptyset$}
\Local{$\emptyset$}
\BlankLine
  \eIf{$n \neq 0$}{$accu \leftarrow \mbox{factorielleRec($n-1$)} \times n$\;}{$accu \leftarrow 1$\;}
\end{algo}

\begin{algo}{vectorAdd128}
\Input{$A \in \Integer^{128}, B \in \Integer^{128}$}
\Output{$C \in \Integer^{128}$}
\Global{$\emptyset$}
\Local{$i \in \Integer$}
\BlankLine
  \For{$i \leftarrow 0$ \KwTo $127$} {
    $C_{i} \leftarrow A_{i} + B_{i}$ \;
  }
\end{algo}

Du blabla pas pris en compte dans la compilation...

\begin{algo}{main}
\Input{$\emptyset$}
\Output{$b \in \Boolean$}
\Global{$\emptyset$}
\Local{$fac \in \Integer$}
\BlankLine
$b \leftarrow \false$ \;
\eIf{$b \vee \true$}{
  $fac \leftarrow \mbox{factorielle($10 \times 1$)}$ \;
  $\mbox{printInt($fac$)}$ \;
}{$\mbox{printInt($-1$)}$ \;}
\end{algo}



\end{document}

